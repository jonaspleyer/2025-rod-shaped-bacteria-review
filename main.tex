\documentclass{article}

\usepackage{arxiv}

\usepackage[utf8]{inputenc} % allow utf-8 input
\usepackage[T1]{fontenc}    % use 8-bit T1 fonts
\usepackage[hidelinks]{hyperref}       % hyperlinks
\usepackage{url}            % simple URL typesetting
\usepackage{booktabs}       % professional-quality tables
\usepackage{amsmath,amssymb,amsthm}
\usepackage{amsfonts}       % blackboard math symbols
\usepackage{nicefrac}       % compact symbols for 1/2, etc.
\usepackage{makecell}
\usepackage{microtype}      % microtypography
\usepackage{mathrsfs}
\usepackage{float}
\usepackage{graphicx}
\usepackage{doi}
\usepackage{acronym}
\usepackage{listings}
\usepackage{tabularx}
\usepackage{tikz}
\usepackage[dvipsnames]{xcolor}

\usetikzlibrary{trees}

\def\checkmark{\tikz\fill[scale=0.4](0,.35) -- (.25,0) -- (1,.7) -- (.25,.15) -- cycle;}
\def\cross{\tikz\draw[scale=0.3, black, line width=0.3mm](0,0) -- (1,1) -- (0.5,0.5) -- (0,1) -- (1,0) -- (0.5,0.5);}

\newacro{abm}[ABM]{Agent-Based Model}
\newacro{cabm}[CABM]{Cellular Agent-Based Model}
\newacro{ca}[CA]{Cellular Automaton}
\newacro{ib}[IB]{Individual-Based}
\newacroplural{ca}[CA]{Cellular Automata}
\newacro{ecoli}[\textit{E.coli}]{\textit{Escherichia coli}}
\newacro{bsubtilis}[\textit{B.subtilis}]{\textit{Bacillus subtilis}}
\newacro{spombe}[\textit{S.pombe}]{\textit{Schizosaccharomyces pombe}}

\newcommand{\todo}[1]{\colorbox{WildStrawberry}{\textcolor{white}{#1}}}

\newcommand{\R}{\mathbb{R}}

\title{Mechanical Models of Rod-Shaped Bacteria - From Single-Cell Modeling to Collective Phenomena}

%\date{September 9, 1985}	% Here you can change the date presented in the paper title
%\date{} 					% Or removing it

\author{
    \href{https://orcid.org/0009-0001-0613-7978}{
        \includegraphics[scale=0.06]{orcid.pdf}
        \hspace{1mm}Jonas Pleyer
    }
    \thanks{
        \href{https://jonas.pleyer.org}{jonas.pleyer.org},
        \href{https://cellular-raza.com}{cellular-raza.com}
    }\\
	Freiburg Center for Data-Analysis and Modeling\\
	University of Freiburg\\
	\texttt{jonas.pleyer@fdm.uni-freiburg.de} \\
	%% examples of more authors
	\And
	\href{https://orcid.org/0000-0002-6371-4495}{
        \includegraphics[scale=0.06]{orcid.pdf}
        \hspace{1mm}Christian Fleck
    }\\
	Freiburg Center for Data-Analysis and Modeling\\
	University of Freiburg
}

% Uncomment to remove the date
%\date{}

% Uncomment to override  the `A preprint' in the header
\renewcommand{\headeright}{Preprint}
%\renewcommand{\undertitle}{Technical Report}
\renewcommand{\shorttitle}{Mathematics of rod-shaped bacteria}

\usepackage{enumitem}
\setlist{nolistsep}

%%% Add PDF metadata to help others organize their library
%%% Once the PDF is generated, you can check the metadata with
%%% $ pdfinfo template.pdf
\hypersetup{
pdftitle={Mathematics of rod-shaped bacteria},
pdfsubject={q-bio.NC, q-bio.QM},
pdfauthor={Jonas Pleyer, Christian Fleck},
pdfkeywords={},
}

% Define definition, example, lemma, proof and theorem.
\newtheorem{definition}{Definition}[section]
\newtheorem{example}[definition]{Example}
\newtheorem{lemma}[definition]{Lemma}
\newtheorem{corollary}[definition]{Corollary}
\newtheorem{theorem}[definition]{Theorem}

% Change numbering of equations
% \numberwithin{equation}{section}

% MAKE TITLES IN THEOREMS BOLD
\makeatletter
\def\th@plain{%
  \thm@notefont{}% same as heading font
  \itshape % body font
}
\def\th@definition{%
  \thm@notefont{}% same as heading font
  \normalfont % body font
}
\makeatother

\begin{document}
\maketitle

% TABLE OF CONTENTS
% Remove this before submission

%###################################################################################################
\begin{abstract}
\end{abstract}

% keywords can be removed
\keywords{Single-Cell \and Mechanics \and Rod-Shaped \and Bacteria \and Mathematics}

\textbf{POSSIBLE TITLES}
\begin{itemize}
    \item Mechanical Models of Rod-Shaped Bacteria - From Single-Cell Modeling to Collective Phenomena
    \item Mathematical Modeling of Rod-Shaped Bacteria - From Single-Cell Mechanics to Collective Phenomena
    \item Mathematical Models of Rod-Shaped Bacteria - From Single-Cell Analyses to Collective Phenomena
\end{itemize}

%###################################################################################################
\section{Introduction}

The last decades have fundamentally challenged and changed our understanding of how bacteria retain
their shape.
Rod-shaped bacteria can grow by extending their circular part, or by inserting new material
at the tip of the rod.
Species such as \ac{ecoli} and ac{bsubtilis}~\cite{Errington2020} fall under the first category
while \ac{spombe} represents the latter.
The material is inserted in small bursts on the nanoscale in forms of patches, bands or
hoops~\cite{DePedro2003}.
Despite their differences in wall-size, \ac{bsubtilis} and \ac{ecoli} follow growth rules which are
comparable with respect to the extension of the rod \cite{Chang2014}.

It was long believed that bacteria lack cytoskeletal filaments but with the end of the last
century, it was discovered that the MreB protein fills this role since it "[..] forms an actin-like
cytoskeleton in bacteria [..]" \cite{Erickson2001} (supplement: [p. 1]).
It can polymerize and form filaments which behave similar to actin microfilaments \cite{Dersch2020}.
Studies of its crystal structure further showed the similarity to actin~\cite{vandenEnt2001} and the
MreB, MreC and MreD proteins were identified as homologues of actin \cite{Lowe2017_lj}.\
\cite{Jones2001} studied the species of different bacterial sub-kingdoms which have the MreB gene.
They concluded: "Many of the organisms are rod shaped, similar to B. subtilis and E. coli.
Organisms with more complex shapes, including curved, filamentous, and helical bacteria, were all
abundantly represented." \cite{Jones2001} (supplement: [p. 7]).
Furthermore \cite{Wachi1987} found that mutants of \ac{ecoli} which create defective MreB proteins will be
spherical instead of their natural rod-shaped form.
Our colleagues from Freiburg have validated many of these observations and further showed that MreB
travels along the axis of elongation with a velocity that depends on the size of the filament
\cite{Olshausen2013}.

This raises the question whether a rod shape serves any particular purpose.
It can be argued that the earliest cells consisted of "[..] rods and filaments with cocci being
derived and degenerate forms." \cite{Young2006}.
The importance of shapes on the function of cells is highlighted by~\cite{Takeuchi2005} who grew
\ac{ecoli} in very narrow microchambers and thus permanently altered their shape.
This resulted in modified motion of the cells, some unchanged (helical and short crescent cells),
some dysfunctional (coiled cells).
It is important to note that this experiment controlled only the shape of bacteria without altering
any additional biochemical or genetic functions.

\section{Biological Principles}
\subsection{How Bacteria grow and maintain their Shape}
\subsection{Bending Stiffness}
\subsection{Physical Interaction between individual Bacteria and Surfaces}
\subsection{Spatial Effects and Self-Organization}

\section{Mathematical \& Computational Frameworks}
\subsection{(Required) Modeling Aspects}

\begin{table}[H]
    \centering
    \def\arraystretch{1.3}
    \begin{tabularx}{\textwidth}{c l X}
        &\textbf{Aspect} & \textbf{Description}\\
        \toprule
        &\textbf{(C) Cellular}\\
        \midrule
        (1) & Rod-Shaped Mechanics &
            Rod-shaped bacteria are flexible rods which are able to freely move around (off-lattice
            approach) \cite{Takeuchi2005,Ursell2014,Amir2014_2}.\\
        (2) & Growth &
            Cells grow exponentially by inserting new material either along the circular part of the
            rod or at the tip~\cite{Robert2014,Takeuchi2005}.\\
        (3) & Differentiation &
            \textit{B.subtilis} is known to differentiate into matrix-producing and
            surfactin-producing cells \cite{vanGestel2015,Lpez2010}.\\
        (4) & Division &
            The formation and continuation of van Gogh bundles is driven by cell division
            \cite{vanGestel2015}.
            Bacteria can form multilayers during their growth phase \cite{Duvernoy2018}.\\
        (5) & Variable Parameters &
            Parameters for individual cells are not fixed values but rather taken from a
            distribution \cite{Koutsoumanis2013}.\\
        &\textbf{(CC) Cell-Cell Interactions}\\
        \midrule
        (6) & Adhesion &
            Bacteria adhere to each other at longer distances and attach when in close contact
            \cite{Verwey1947,Trejo2013}.
            The interaction between rods can be polarized \cite{Duvernoy2018}.\\
        (7) & Friction &
            Friction between cells \cite{Grant2014} can be asymmetrical \cite{Doumic2020}.\\
        &\textbf{(DC) Domain-Cell Interactions}\\
        \midrule
        (8) & External Forces &
            Bacteria tend to stick to surfaces \cite{vanLoosdrecht1989}.
            The extracellular gel exerts a force onto the cells \cite{Grant2014}.\\
        (9) & Extracellular Reactions &
            Bacteria can take up nutrients or possible secrete/take up signalling molecules
            \cite{Li2025}.\\
        \bottomrule
    \end{tabularx}
    \label{table:simulation-aspects-supplement}
    \caption{TODO}
\end{table}

\subsection{Computational Modeling Frameworks}
\subsubsection{Many general-purpose Frameworks lack explicit support for Rod-Shaped Bacteria}
\subsubsection{Varying Support for Rod-Shaped Bacteria}
\subsection{Purpose-Built Models}

\section{Discussion}
\section{Conclusion}

\bibliographystyle{IEEEtran}
\bibliography{references}

\section{Supplementary Material}

\end{document}
