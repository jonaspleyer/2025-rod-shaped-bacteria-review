
\paragraph{Cell Wall}
\begin{itemize}
    \item \cite{Cava2014} main cell envelope of bacteria is a mesh-like macromolecule of glycan strands that are crosslinked via short peptides. Thus  peptidoglycan \ac{pg}
    \item \cite{vanTeeffelen2018} \ac{pg} PG meshwork mechanically resists the high turgor pressure and gives the cell its specific cell shape
    \item \cite{Cochrane2020} \ac{pg} chemically unique (to bacteria) and essential for cell survival. Therfore it's synthesis and turnover is targeted by a range of antibacterial treatments
        Different mechanism based on use of polar growth zones derived from the division machinery (\ac{pg} and FtsZ)
    \item \cite{Amir2014} bacterial cell shape is physically determined by the \ac{pg} cell wall but not sufficient for maintaining rod-shape on its own, need more effects
\end{itemize}

\paragraph{Cytoskeleton}
\begin{itemize}
    \item It was long believed that bacteria lack cytoskeletal filaments but with the end of the last century, it was discovered that the MreB protein fills this role since it "[..] forms an actin-like cytoskeleton in bacteria [..]" \cite{Erickson2001}.
    \item \cite{Dersch2020} It can polymerize and form filaments which behave similar to actin microfilaments 
    \item \cite{Lowe2017_lj} Studies of its crystal structure further showed the similarity to actin~\cite{vandenEnt2001} and the MreB, MreC and MreD proteins were identified as homologues of actin 
    \item \cite{Bratton2018} trans-membrane protein RodZ can modulate MreB density. It has a direct effect on MreB curvature preference
\end{itemize}

\paragraph{Interplay}
\begin{itemize}
    \item \cite{White2012} \cite{López2006} MreB adopts a spiral-like or banded pattern along the length of the cell coordinating \ac{pg} synthesis  
    \item  \cite{Shi2018} MreB filaments move circumferentially around rods \cite{Garner2021} of any width, resultant \ac{pg} synthesis around the rod circumference is stabilizing rod shape against the internal pressure
    \item \cite{Shi2018} MreB is a cytoskeletal organizing element it's localization also directs cell growth to locally change cell shape 
    \item \cite{vanTeeffelen2018} MreB filaments are attracted to regions of negative Gaussian curvature and excluded from regions of positive Gaussian curvature, such as the cell poles, contribtuting to a rod-shape
    \item \cite{DEscobar2011} MreB polymers actively restrict and/or control the mobility of cell wall elongation complexes
    \item \cite{Olshausen2013} got good resolution of MreB dynamics with the help of \ac{tirf}. Their mathematical model confirms that MreB acts as mechanical organizer, coupling the cytoskeleton with cell wall synthesis to maintain rod shape.
\end{itemize}

\paragraph{Growth}
\begin{itemize}
    \item \cite{Chang2014} cell elongation is comparable for gram-negative and gram-positive bacteria (\ac{bsubtilis} and \ac{ecoli}) despite their wall thickness
    \item \cite{Cooper1991} \cite{Wang2010_2} \ac{pg} synthesis is spatially and temporally regulated. During elongation, cells grow at a constant diameter, expanding surface mainly along the sidewall. During constriction polar synthesis is initiated forming new polar caps
    \item \cite{Billaudeau2017} Growth is MreB-dependent, maintaining constant diameter
    \item \cite{Daniel2003} 2 ways maintaining rod shape: with/without MReB: A few rod-shaped bacteria have no MreB system. MreB-guided lateral wall growth or polar elongation
    \item \cite{DePedro2003} new \ac{pg} is inserted in discrete patches along the lateral sidewall
    \item 
    \item \cite{Si2015} Bacterial morphogenesis and \ac{pg} synthesis are mechanically sensitive, showing that growth patterns are adaptive and coupled to physical forces.
    \item \cite{Chatterjee1988} spatial variation of mechanical stress in the cell also contributes to growth patterns, shape maintenance, and division site selection.
    \item \cite{Billaudeau2017} wall thickness influence elongation dynamics (species-specific coordination) (details if needed: in \ac{bsubtilis} and \ac{ecoli} \ac{pg} assembly, spatial pattern, wall thickness, and machinery organization differ, leading to distinct elongation dynamics )
\end{itemize}

\paragraph{Maintaining/Loosing Rod-shape}
\begin{itemize}
    \item \cite{Jones2001} MreB is responsible for cell width regulation and Mbl is responsible for linear axis regulation. Beyond rod-shaped bacteria.
    \item \cite{Lleo1990} shape is balanced by two reaction i) wall elongation ii) septum formation. cocci shaped lost i) or are induced by mutations in i)
    \item e.g. for lacking i): \ac{a22} treated cells \cite{IWAI2002}, \textit{rodZ} deletion mutants \cite{Shiomi2008} and \ac{mre} mutants \cite{Wachi1987} were found to induce spherical cells shape indicating that cell shape maintenance involves proteins encoded by the \ac{mre} like MreB or interacting with them
    \item extension to point before: specifying what \ac{a22} is. \cite{Bean2009} \ac{a22} binds directly to nucleotide-free MreB, due to its high affinity to the nucleotide-binding pocket of MreB. Thereby disrupting MreB assembly
    \item \cite{Garner2021} MreB filaments move circumferentially around rods of any width, but filament motion was isotropic in spherical cells, moving in all directions
    \item Switch in or \denovo \ac{pg} synthesis is sufficient to regenerate the cell wall and restore a rod-shape morphology \cite{Huan2021} spherical to rod-shape  \cite{Kawai2014} L-form to rod-shape
    \item \todo{not yet formulated}
    \item \cite{Wang2010, Wang2010_protocol} MreB and \ac{pg} contributes to nearly equal parts to the stiffness of a cell
    \item \cite{Amir2014_2} elastic deformations after pulse-like force do not change the shape permanently. Plastic deformations occur adapting to external bending forces causing the cell walls to grow differentially leading to persisting plastic deformations
\end{itemize}

\paragraph{Cell division}
\begin{itemize}
    \item \cite{Bramkamp2009} Two inhibitory systems regulates the assembly FtsZ \cite{Oliva2004}: i) Min system prevents division near poles and ii) nucleoid occlusion prevents cell division over the nucleoids
    \item adding to first point \cite{Koch1995} \cite{Chatterjee1988} central FtsZ-ring assembly is favored by mechanical forces. Its regulation is thereby a combination of mechanical cues and biochemical regulation
    \item FtsZ filaments are forming a ring (Z-ring) at division site \cite{Li2007} generating constriction force for division. The ring tension is regulating \ac{pg} synthesis \cite{Lan2007}
    \item \cite{DenBlaauwen2008} divisome or septosome switches \ac{pg} synthesis from dispersed to a concentrated local to from new cell poles
    \item \cite{Woldringh1987} \cite{Cooper1991} surface synthesis rate increase at the start of constriction, elongation and pole formation use distinct, spatially regulated \ac{pg} insertion modes, supporting the idea of separate elongation and division machineries (divisome/septosome)
    \item \cite{Kruse2005} \cite{Gitai2005} MreB contributes in forming a bacterial mitotic-like machine and control Bacterial DNA segregation
    \item \cite{Karczmarek2007} DNA and origin region segregation are not affected by the transition from rod to sphere after inhibition of Escherichia coli MreB by \ac{a22} \todo{do we really need this extra information?}
    \item \cite{denBlaauwen2018} bacterial morphogenetic is plastic and can be reprogrammed changing growth and division axis \todo{@toqi do we really need this for modelling EMERGENT phenomena?}
    \item \cite{Stewart2005} despite symmetric division aging and mortality can occur by asymmetric inheritance of cellular damage (older “mother” cell poles accumulate damage or growth defects over generations)
\end{itemize}z

\paragraph{Rod-shape vs. other shape}
\begin{itemize}
    \item \cite{Young2006} shape is an evolutionary trait which influence nutrient uptake, cell division and segregation, adhesion, passive and active motility, predator avoidance, and cellular differentiation.
    \item e.g. for point before: \cite{Takeuchi2005} changed \ac{ecoli} morphologies without altering any additional biochemical or genetic functions and automatically altered cell motility
    \item Other shapes often request other or additional cell-wall-machineries \cite{Zapun2008} or cytoskeletal components like crescentin which are bacterial equivalent to intermediate filaments (IFs\cite{Ausmees2003}
    \item to conclude: \cite{Jones2001} "We therefore suggest that a sphere is the default shape taken by cells in the absence of an active, mreB-dependent shape determining system, and that more complex shapes are usually determined in part by \textit{mreB} systems". \cite{Young2006} "It can be argued that the earliest cells consisted of "[...] rods and filaments with cocci being
\end{itemize}

\paragraph{Taxi}
\begin{itemize}
    \item \cite{Krell2011} multiple diverse taxis systems integrating chemical (chemotaxis), light (phototaxis), oxygen (aerotaxis), temperature (thermotaxis), and magnetic fields (magnetotaxis) stimuli. Bacterial taxis are versatile and highly evolutionary adapted to the species environment.
    \item \cite{Nikita2009, Bourret2002} Chemotaxis, is the most common and best studied form of taxis. It is a receptor-mediated signaling network that integrates external chemical cues to modulate flagellar motion and control movement direction.
\end{itemize}

\paragraph{Nutrient uptake/distribution}
\begin{itemize}
    \item \cite{Davies2021, Niederweis2008, Tanaka2018} Bacterial nutrient uptake is mediated by highly selective transporter systems (e.g., ABC transporters, TRAP systems, and PTS) that import sugars, amino acids, ions, carbon, nitrogen, iron, and lipids; bacteria can also modulate cell envelope permeability and secrete enzymes to facilitate nutrient acquisition
    \item \cite{Li2025} matrix-producing improve access to nutrients, supporting growth under nutrient-limited conditions
    \item \cite{Watteaux2015} Motile chemotactic bacteria have an enhances nutrient uptake rate. 2.2 times faster than non-motile 
    \item \cite{Caron1994} Bacteria acquire significant nitrogen and phosphorus from dissolved inorganic nutrients, competing with phytoplankton and influencing the microbial loop; overall contribution to ocean nutrient uptake is poorly understood
    \item \cite{Velicer2009,Rory2015} a diverse part of bacterial are predators and found in various environments, plaing a significant role in microbial community structure and dynamics, with different modes of predation and independent evolutionary origins
\end{itemize}

\paragraph{Movement (swimming, gliding, recipe for swarming)}
\begin{itemize}
    \item \textbf{swimming}
    \item \cite{Shen2022} bending stiffness of the flagellar filament support propulsion and reorientation during swimming, it does not vary
    \item \cite{Kong2014} swimming motion of rod-shaped magnetotactic bacteria is regulated by shape and magnetism
    \item \cite{Alberti1990} swimming cells are short, mononucleate, flagellated for planktonic motility. Can differentiate to swamers, when growing on and surface
    \item \textbf{swarming}
    \item \cite{Harshey2015} compared to swimming which direct a single through chemotaxis to find nutrients and avoid toxic environments swarming is a collective motion suppressing chemotaxis and using the dynamics of their collective motion to continuously expand and acquire new territory
    \item \textbf{gliding}
    \item \cite{ContrerasM2024} gliding relies on a outer-membrane complex that attaches the substrate at fixed sites of focal adhesion enabling directional gliding motility
    \item \cite{Chang2016}type IV pili generates retraction force for twitching motility, pulling the cell across surfaces 
    \item \cite{Shah2022} model shows that non-Newtonian slime behavior enhances gliding motility by optimizing force transmission and energy use
    \item \cite{Shrivastava2015} \ac{flavo} and \textit{Myxococcus xanthus} are gliding rod-shpaed flagelaten-free spicies. Their gliding relies on rotary motors driving mobile surface filaments, providing continuous propulsion analogous to a snowmobile
\end{itemize}

\paragraph{Cell-Cell Forces/Adhesion/Friction}
\begin{itemize}
    \item \cite{vanLoosdrecht1989, Hori2010} Bacterial adhesion is driven by physicochemical interactions: van der Waals, electrostatic, hydrophobic, and acid-base forces and it's strength is determined by both surfaces properties (charge, hydrophobicity, roughness) determine
    \item \cite{BrettFinlay2014, Berne2018} Adhesins, pili/fimbriae, and surface proteins mediate the binding to \ac{ecm} components/ adhesion is a highly heterogeneous and dynamic process. Binding to a host \cite{Beachey1981, Vaca2019}
    \item \cite{Vaca2019,Beachey1981} Gram-negative bacteria (eg. \ac{ecoli}) express outer membrane proteins (adhesins) which allow them to attach to various hosts or cadherins to generate mecnical links between neighboring cells \cite{Whitfield2006}
    \item Type IV pili \cite{Ellison2021, Maier2015} or flagella in general \cite{Haiko2013} are more than force-generating structures that coordinate bacterial movement, they play an important role in the direct adhesion, and environmental interaction, e.g. pulling cells together and promote microcolony formation.
    \item some bacteria can fuse to each other when they are long enough at close distance \cite{Kudryashev2011}
    \item \cite{Alric2022} Crowding and cell-cell forces act together to constrain growth, linking intracellular density with external mechanical stress
    \item \cite{Badyaev2025} Mechanical interactions during jamming reshape cellular behavior and gradients, enabling long-term adaptive diversification
    \item \todo{friction?}
\end{itemize}

\paragraph{Adaptation}
\begin{itemize}
    \item \cite{Bertrand2019} bacteria encountering new nutrients have an adaptive phase: lag-phase in which there is now division. While this dormant stage metabolic adaptation and macromolecular synthesis occur to prepare  for rapid cell growth and robust cell division
    \item \cite{Senkei2007} Bacterial populations exhibit heterogeneous, asynchronous responses in lag times, swimming speeds, or surface motility activation for adaptive advantage
    \item \cite{Barrangou2007} CRISPR provides adaptive resistance against bacteriophages \cite{Duckworth2002} in prokaryotes (\cite{Borges2017} some protein families have anti-CRISPR function)
\end{itemize}

\paragraph{Quorum Sensing}
\begin{itemize}
    \item \cite{Stephen2007} Bacteria produce and secrete small signaling molecules called autoinducers called Quorum sensing. Coordinate gene expression collectively, enabling adaptive group behavior
    \item \cite{MorenoGmez2023} Quorum sensing allows bacteria to collectively process information enhancing decision accuracy through population-level coordination
    \item \cite{Hense2007, Boyer2009}  Quorum sensing not only within single species, but also interspecies and cross-kingdom (e.g., bacteria-host, bacteria-virus). \cite{Liu2025} Highlights some applications in biomedical and environmental technologies 
    \item \cite{Solano2014} Quorum sensing plays a key role in mature biofilms for e.g transition from a sessile community to a free-living state
\end{itemize}

\paragraph{Differentiation}
\begin{itemize}
    \item \textbf{swarming}
    \item \cite{Harshey1994} swarming is active and rapid surface motility. Swarmer cell is are usually long, multinucleate, hyperflagellated bacterial cell
    \item Swarm-cell have elevated resistance to multiple antibiotics
    \item \textbf{sporilation}
    \item some bacterial can have lifecycle like Myxococcus xanthus \cite{Jose2016}  \ac{bsubtilis} \cite{Branda2001} for e.g task division and environmental adaptation
    \item \cite{Licking2000}  Myxococcus xanthus have starvation-dependent and starvation-independent pathways of sporulation and produce structurally different spores
    \item \cite{Barák2019} \ac{bsubtilis} can divide asymmetrically during sporulation one cell differentiation process 
    \item a variety of bacteria can form spores in order to survive under unfavorable environmental conditions. Germination, these spores restore vegetative morphology
    \item \textbf{fruiting bodies}
    \item \cite{Huan2021} Myxococcus xanthus sporulat and make fruiting bodies.
    \item \cite{Licking2000, Jose2016, Branda2001} certain bacteria from fruiting bodies, a multicellular aggregation of differentiated cells, where spore formation is preferred
    \item \textbf{task division}
    \item \cite{Lopez2010} Extracellular signalling in \ac{bsubtilis} orchestrates diverse, co-existing cell fates via specific signal-sensor-regulator cascades, enabling a community of genetically identical cells to specialize into distinct roles.
    \item \cite{Flemming2016} cellular differentiation/task division in biofilms (matrix production, motility, sporulation)

\end{itemize}

\paragraph{Biofilms}
\begin{itemize}
    \item \cite{Flemming2016} “Biofilms: an emergent form of bacterial life” names emergent phenomena: Enhanced stress tolerance, Collective resource capture, Spatial organization and division of labor, Self-structuring and architecture formation
    \item \cite{Wisnu2022} Mixed-species biofilms have enhanced survival, mediated by antimicrobial-resistant keystone species
    \item \cite{Rendueles2012} Competition in biofilms has a key role. Rivality in surface-colonisation leads to strategies that target adhesion, signalling and matrix dynamics allowing mixed-species biofilms to regulate neighbour entry and community structure
    \item \cite{Dunne2002} Bacterial adhesion is a dynamic, regulated process that underlies biofilm formation and persistence on diverse surfaces
    \item \cite{Ong1999, vanLoosdrecht1989, Hori2010} initial colonization and biofilm potential is governed by the bacterial features and biomaterial properties
    \item \cite{Duvernoy2018} shaping early biofilm morphogenesis by asymmetric adhesion of rod-shaped bacteria to surfaces early biofilm transition architecture from 2D to 3D structure when confinement and under mechanical forces.
    \item \cite{Grant2014} 
\end{itemize}